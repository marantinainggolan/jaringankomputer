\documentclass[conference]{IEEEtran}
\usepackage{graphicx}

% path gambar
\graphicspath{{./picture/}}

% JUDUL %%%%%%%%%%%%%%%%%%%%%%%%%%%%%%%%%%%%%%%%%%%%%%%%%%%%%
\title{Analisis Kekuatan Sinyal Menggunakan InSSIDer}

% PENULIS %%%%%%%%%%%%%%%%%%%%%%%%%%%%%%%%%%%%%%%%%%%%%%%%%%%
\author{Andrian Syah\IEEEauthorrefmark{1}, Hani Khairiyah\IEEEauthorrefmark{2}\\
\textit{Fakultas Teknologi Informasi}\\
\textit{Institut Teknologi Batam}\\
Batam, Indonesia\\
Email: \{\IEEEauthorrefmark{1}1922009, \IEEEauthorrefmark{2}1922001\}@student.iteba.ac.id}

\begin{document}

% untuk mengeluarkan judul dan author
\maketitle

% ABSTRAK %%%%%%%%%%%%%%%%%%%%%%%%%%%%%%%%%%%%%%%%%%%%%%%%%%
\begin{abstract}
Implementasi jaringan wireless (atau umum disebut sebagai jaringan WiFi) telah diatur oleh standar IEEE 802.11. Jaringan WiFi digunakan untuk menghubungkan berbagai perangkat dan berbagi data. Seringkali disuatu daerah penggunaan WiFi sangat banyak sehingga mengganggu satu sama lain. Analisis Sinyal WiFi adalah kegiatan yang melibatkan analisis lalu lintas WiFi. Pada tugas ini mahasiswa diminta untuk melakukan analisis sinyal WiFi menggunakan alat analisa yang disebut InSSIDer. Mahasiswa diharapkan dapat mengumpulkan informasi tentang WiFi dan menganalisis paket yang ditangkap untuk memahami sifat dan kinerja WiFi di Lingkungan sekitar (disarankan untuk mencari tempat yang menyediakan hotspot yang banyak). Perhatikan pula bila suatu lingkungan terdapat jaringan WiFi yang banyak akan ada saluran tumpang tindih dan mendapatkan kekuatan sinyal yang lebih sedikit.
\end{abstract}

% kata kunci %%%%%%%%%%%%%%%%%%%%%%%%%%%%%%%%%%%%%%%%%%%%%%%%
\begin{IEEEkeywords}
signal strength
\end{IEEEkeywords}

% pendahuluan %%%%%%%%%%%%%%%%%%%%%%%%%%%%%%%%%%%%%%%%%%%%%%
\section{Pendahuluan}
tuliskan statement sisikan referensinya, statement lain sisipkan pula referensinya. dan seterusnya. Dalam kehidupan sehari-hari, manusia tidak luput dari ben-cana yang dapat membahayakan dan membuat kerugian. Keru-gian  tersebut  dapat  berupa  material  maupun  secara  fisik  danmental.  Dalam  pesatnya  kemajuan  hidup  manusia,  manusiamemikirkan  cara  untuk  mengurangi  kerugian  tersebut  den-gan  membuat  suatu  pemikiran  yang  berkaitan  dengan  caramenghindari  bahaya  yang  basisnya  adalah  insting  alamiahdari manusia sebagai makhluk hidup. Dengan menggabungkanteknologi, manusia menciptakan pemikiran jalur evakuasi danjalur  terefektif  dan  terdekat  yang  berbentuk  suatu  algoritmayang  dapat  dengan  mudah  melakukan  perhitungan  dan  pem-rosesan tanpa membutuhkan waktu yang lama~\cite{hata1980mobile}.

\section{Related Work}
tuliskan statement sisikan referensinya, statement lain sisipkan pula r

\section{Design}
tuliskan statement sisikan referensinya, statement lain sisipkan pula. lihat pada gambar~\ref{fig1}

\begin{figure}[ht!]
    \def\svgwidth{\columnwidth}
    \input{picture/user.pdf_tex}
    \caption{Top Global Threat Rank}
    \label{fig1}
\end{figure}

\section{Result and Discussion}
tuliskan statement sisikan referensinya, statement lain sisipkan pula r

\section{Conclusion}
tuliskan statement sisikan referensinya, statement lain sisipkan pula r

% referensi %%%%%%%%%%%%%%%%%%%%%%%%%%%%%%%%%%%%%%%%%%%%%%%%
\bibliographystyle{IEEEtran}
\bibliography{referensi}


\end{document}